\documentclass{ctexart}
\title{Algorithm Homework3}
\author{PB18111704 Zhu Enzuo}
\date{\today}
\usepackage{algorithm} %ctan.org\pkg\algorithms
\usepackage{algpseudocode}
\usepackage{amsmath}
\usepackage{tikz}
\begin{document}
\maketitle
\subsection{Prob1}
\begin{algorithm}
    \begin{algorithmic}[1]
    \Procedure{Makeset}{$*A,v$}
        \State A->key=v;
        \State A->parent=A;
    \EndProcedure
    \end{algorithmic}
\end{algorithm}
\begin{algorithm}
    \begin{algorithmic}
        \Procedure{Findset}{$*A$}
            \If{A->parent->key==A->key}
                \State $return A$
            \Else
                \State $A->parent\gets Findset(A->parent)$
                \State $return A$
            \EndIf
        \EndProcedure
    \end{algorithmic}
\end{algorithm}
\begin{algorithm}
    \begin{algorithmic}
        \Procedure{Union}{$*A,*B$}
            \State $Aset=Findset(A)$
            \State $Bset=Findset(B)$
            \State $Bset->parent=Aset$
        \EndProcedure
    \end{algorithmic}
\end{algorithm}
\subsection{Prob2}

状态:pack[i][w] 表示从前i个商品中选择物品装入背包有至少一种方法使得背包中商品总重量为w

初始状态:pack[i][0]=1

转移:pack[i][j]=$pack[i-1][j-item_i.weight]$

最终结果:$ans=max_j(pack[n][j]=1)$

\end{document}
