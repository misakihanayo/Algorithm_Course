% LaTeX file for resume
% This file uses the resume document class (res.cls)

\documentclass[margin, 10pt]{res}
\usepackage{url}
\usepackage{enumitem}
\usepackage[colorlinks=true,urlcolor=blue]{hyperref}

\usepackage{fancyhdr}

\pagestyle{fancy}
\fancyhf{}
\cfoot{Page \thepage}
\renewcommand{\headrulewidth}{0pt}
% the margin option causes section titles to appear to the left of body text
%\textwidth=5.6in % increase textwidth to get smaller right margin
%\usepackage{helvetica} % uses helvetica postscript font (download helvetica.sty)
%\usepackage{newcent}   % uses new century schoolbook postscript font

\begin{document}

\setlist[itemize]{leftmargin=*}
\name{Enzo Zhu\\[16pt]} % the \\[12pt] adds a blank line after name
\address{{\bf Present Address} \\ Dept. of Computer Science\\ USTC\\
        230026 Hefei, P.R.China  }
\address{{\bf Contact Info} \\ (+86)18618154314 \\
         zhuez1819@mail.ustc.edu.cn }

\begin{resume}

\section{Current \\ Position}
Student at the Department of Computer Science, University of Science and Technology of China (USTC).

\section{Interests}
My research interests lie at the intersection of verification techniques in computer systems or softwares.

\section{Education}
{\bf Hong Kong University} (HKU), Hong Kong  \hfill Summer 2021 \\
Internship\\
Worked on an implemention of trusted accelerator execution in ARM trustzone \\
Advisor: Heming Cui

{\bf University of Science and Technology of China} (USTC), Hefei, China \hfill 2018 - 2022 \\
B.S. in Computer Science

\section{Awards}
Third Prize in National Computer System Contest in Compiler Designing \hfill 08/2021 \\
Second Prize in National Computer System Contest in Chip Designing \hfill 08/2020

\section{Work \\ Experience}
Part-time Computer Programming TA in USTC \hfill Sep-Dec 2020 \\
Part-time Principles and Techniques of Compiler TA in USTC \hfill Sep-Dec 2021

\section{Project \\ Experience}
{\bf Bazinga Compiler} \url{https://github.com/misakihanayo/bazinga_compiler} \hfill 08/2021 \\
I worked in a team on a compiler to translate code from sysc language to arm-v8 assembly code.
Within this project, I implemented the lexer, the parser, part of the middle IR, several optimization passes, part of the lower IR and the code generator.

{\bf ustc-nscscc-2020-1} \url{https://github.com/misakihanayo/ustc-nscscc-2020-1} \hfill 08/2020 \\
I worked in a team on a 5 stage MIPS Chip. Within this project, I implemented the MEM stage,
 the exception module, the TLB and the instruction and data cache.

{\bf FPGAOL} \hfill 07/2019-12/2019 \\
I worked in a team on a web system to provide online FPGA support for the collage digital circuits labs.


\end{resume}
\end{document}
