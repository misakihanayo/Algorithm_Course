% LaTeX file for resume
% This file uses the resume document class (res.cls)

\documentclass[margin, 10pt]{res}
\usepackage{url}
\usepackage{enumitem}
\usepackage[colorlinks=true,urlcolor=blue]{hyperref}

\usepackage{fancyhdr}

\pagestyle{fancy}
\fancyhf{}
\cfoot{Page \thepage}
\renewcommand{\headrulewidth}{0pt}
% the margin option causes section titles to appear to the left of body text
%\textwidth=5.6in % increase textwidth to get smaller right margin
%\usepackage{helvetica} % uses helvetica postscript font (download helvetica.sty)
%\usepackage{newcent}   % uses new century schoolbook postscript font

\begin{document}

\setlist[itemize]{leftmargin=*}
\name{Enzo Zhu\\[16pt]} % the \\[12pt] adds a blank line after name
\address{{\bf Present Address} \\ Dept. of Computer Science\\ USTC\\
        230026 Hefei, P.R.China  }
\address{{\bf Contact Info} \\ (+86)18618154314 \\
         zhuez1819@mail.ustc.edu.cn }

\begin{resume}

\section{Interests}
My research interests lie at Programming Language and System, Formal Methods and System Verification. 
\section{Education}

{\bf University of Science and Technology of China} (USTC), Hefei, China \hfill 2018 - 2022 \\
Undergraduate \\
B.S. in Computer Science \\
* Performance: Major 3.02/4.3 or 80.43/100


\section{Awards}
Third Prize in National Computer System Contest in Compiler Designing \hfill 08/2021 \\
Second Prize in National Computer System Contest in Chip Designing \hfill 08/2020

\section{Work \\ Experience}
Part-time Principles and Techniques of Compiler TA in USTC \hfill Sep-Dec 2021 \\
Part-time Computer Programming TA in USTC \hfill Sep-Dec 2020

\section{Project and \\ Research \\ Experience}
{\bf ORMIR} 09/2021 - current \\
I worked with Kai Ma to design a automatic verifier which verify Django project and ORM operations' consistency properties.

{\bf Bazinga Compiler} \url{https://github.com/misakihanayo/bazinga_compiler} \hfill 08/2021 \\
I worked in a team on a compiler to translate code from sysc language to arm-v8 assembly code.
Within this project, I implemented the lexer, the parser, part of the middle IR, several optimization passes, part of the lower IR and the code generator.

{\bf Heterogeneous Isolated Execution in ARM TrustZone} (HKU), Hong Kong  \hfill Summer 2021 \\
Internship\\
I designed a trusted accelerator in ARM trustzone with my advisor Heming Cui, including a PCIe device to 
control the execution zone of a program. This device can distinguish codes from different zone(Trusted or Normal), 
execute them in different zone of the system, and forbid inaccurate access to data.

{\bf USTC-SYS Reading Group} Fall 2020 - current \\
I participanted in the reading group and went through a bunch of up-to-date, state of the art papers,
 and gave a report about the paper "Storage Systems are Distributed Systems (So Verify Them That Way!)", OSDI 20.

{\bf Ustc-Nscscc-2020-1} \url{https://github.com/misakihanayo/ustc-nscscc-2020-1} \hfill 08/2020 \\
I built a 5 stage MIPS Chip in a team. Within this project, I designed the MEM stage,
 the exception module, the TLB and the instruction and data cache. the chip can run a P-mon system on it, and support 
 VGA video output.

{\bf FPGAOL} \hfill 07/2019-12/2019 \\
I worked in a team on a web system to provide online FPGA support for the collage digital circuits labs.


\end{resume}
\end{document}
