\documentclass{ctexart}
\title{Algorithm Homework1}
\author{PB18111704 Zhu Enzuo}
\date{\today}
\usepackage{algorithm} %ctan.org\pkg\algorithms
\usepackage{algpseudocode}
\usepackage{amsmath}
\usepackage{tikz}
\begin{document}
\maketitle
\subsection{Prob1} 聚合分析

$Total\_Operation=N-logN+\sigma_{0<i<logN} 2^i$

$Amortize\_Cost=Total_Operation/N=1-logN/N+(2^{(logN+1)}-1)/N=O(logN)$
\subsection{Prob2} 核算法

\begin{tabular}{|c|c|c|}
\hline
Operation & Actual Cost & Amortized Cost \\
\hline
$i!=2^k$ & 1 & 3 \\
\hline
$i=2^k$ & $2^k$ & 2 \\
\hline 
\end{tabular}

\subsection{Prob3} 势能法 

对于非2的幂的操作,累计2的势能。
这样每当到了2的幂的操作就会正好累计这么多的势能。

所以摊还分析复杂度为O(1)
\subsection{Prob4} 证明如下:

\subparagraph{a}
\[\begin{aligned}
    y_{k_1,k_2,...,k_d} &= \sum_{j_1=0}^{n_1-1}\sum_{j_2=0}^{n_2-1}...\sum_{j_d=0}^{n_d-1} a_{j_1j_2...j_d}
    \omega_{n_1}^{j_1k_1} \omega_{n_2}^{j_2k_2}...\omega_{n_d}^{j_dk_d} \\
    &=\sum_{j_1=0}^{n_1-1}\sum_{j_2=0}^{n_2-1}...\sum_{j_{d-1}=0}^{n_{d-1}-1} \omega_{n_1}^{j_1k_1}
    \omega_{n_2}^{j_2k_2}...\omega_{n_{d-1}}^{j_{d-1}k_{d-1}} \sum_{j_d=0}^{n_d-1} a_{j_1j_2...j_d}\omega_{n_d}^{j_dk_d}\\
    &=\sum_{j_1=0}^{n_1-1}\omega_{n_1}^{j_1k_1}\sum_{j_2=0}^{n_2-1}
    \omega_{n_2}^{j_2k_2}...\sum_{j_d=0}^{n_d-1} a_{j_1j_2...j_d}\omega_{n_d}^{j_dk_d}
\end{aligned}\]
观察最终的求和,其为向量$\vec{a}$先经过d维上的变换然后做d-1维上的变换最终做第1维上的变换。
\subparagraph{b}
由上问的公式,可以看出来求和号顺序的改变对结果没有影响。
\subparagraph{c}
\[\begin{aligned}
Total\_Complexity&=\sum_{i=1}^d n_ilog(n_i) \\
                 &\le n\sum_{i=1}^d n_ilog(n_i)
                 &=nlog(n_1*n_2*n_3*...*n_d)
                 &=O(nlogn)
\end{aligned}\]
\end{document}
